\documentclass[aip,jcp,reprint]{revtex4-1}

\usepackage[utopia]{mathdesign}
\usepackage[utf8]{inputenc} % Unicode input encoding
\usepackage[T1]{fontenc} % Font encoding
\usepackage[english]{babel} % English linebrakes, hyphenation etc.
\usepackage[%
	tracking=true,kerning=true,spacing=true,%
	stretch=10,shrink=10,%
	factor=1100,%
	]{microtype}
	
\usepackage{graphicx}% Include figure files
\graphicspath{ {./figures/} }

\usepackage{multirow}

\usepackage{csquotes}
\usepackage{parskip} % Change paragraph formatting parindent = 0, parskip > 0
\usepackage{bm}% bold math
\usepackage{mathtools}
\usepackage{xspace}
\usepackage{xcolor}

\DeclareMathAlphabet{\mathoms}{OMS}{cmsy}{m}{n}
\DeclareMathOperator{\bigO}{\ensuremath{\mathoms{O}}}

\usepackage{lipsum}
\usepackage{nopageno}

\raggedbottom
%\flushbottom

% Placeholders
\newcommand{\plcite}{{\color{red}\textbf{[???]}}\xspace}
\newcommand{\plinfo}{{\color[rgb]{0.929,0.694,0.125}\textbf{[info]}}\xspace}

\begin{document}
\preprint{AIP/123-QED}

\title[Design of a Pyramid Scheme Lisp-like programming language]{Design of a Pyramid Scheme Lisp-like programming language}

\def\icl{Department of Chemistry, University of Oxford, Inorganic Chemistry Laboratory, Oxford OX1 3QR, U.K.}
\author{Marcin~Konowalczyk}\email{marcin.konow@lczyk.xyz}\affiliation{\icl}

\date{\today}

\begin{abstract}

\lipsum[1]
\end{abstract}

\keywords{a; b; c}

\maketitle

\section{Introduction}

Pyramid Scheme is a variant of the Scheme dialect of Lisp, designed by Conor O'Brien \plinfo \plcite.

\lipsum[1-5]

\section{Language description}
\subsection{Bracket structure}
\subsection{Additional functions}
\subsection{Sharp edges}

\section{Compiler}
\subsection{Abstract syntax tree}
\subsection{Local macro expansion}
\subsection{Optimisation}

\section{Example programs}
\subsection{Bubble sort}
\subsection{Chess engine}

\section{Conclusions}
Program in Pyramid Scheme! Teach your friends! Have them teach \emph{their} friends! Then have those friends teach \emph{their} friends!


\lipsum[11-15]

\section*{Links}
\begin{itemize}
\item psll git
\item pyramid scheme
\item TiO hello world in ps
\item Esolang page?
\item Code Golf page
\end{itemize}

\section*{Acknowledgements}

%\nocite{*}
\bibliographystyle{unsrt}
%\bibliography{psll} %< bib file name

\end{document}